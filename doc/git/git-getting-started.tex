\documentclass[12pt,pdftex]{article}

\usepackage[postscript]{ucs}
\usepackage[utf8x]{inputenc}
\usepackage[T1]{fontenc}
\usepackage{fixltx2e}
\usepackage{graphicx}
\usepackage{longtable}
\usepackage{soul}
\usepackage{textcomp}
\usepackage{marvosym}
\usepackage{amsmath}
\usepackage{wasysym}
\usepackage{latexsym}
\usepackage{amssymb}
\usepackage{amstext}
\usepackage{hyperref}
\usepackage{pifont}
\usepackage{units}
\usepackage[a4paper,margin=20mm,top=28mm,bottom=20mm]{geometry}
\usepackage[bf,sf,small,nonindentfirst]{titlesec}
\usepackage{booktabs}
\usepackage{parskip}
\usepackage{newcent}
\usepackage{helvet}
\usepackage{pxfonts}

\renewcommand{\ttdefault}{txtt} % narrower + nicer tt font
\titleformat{\subsubsection}{\normalfont\itshape}{\thesubsubsection}{.5em}{}
\setlength{\emergencystretch}{1\textwidth} % no overfull hboxes with tt font
\usepackage{listings}
\author{The \textsc{Pencil Code} Team}
\date{19-Jun-2016}
\title{Getting started with Git for the \textsc{Pencil Code}}
\hypersetup{
 pdftitle={Getting started with Git for the Pencil Code},
 pdfauthor={The Pencil Code Team},
 pdfsubject={Version control},
 pdfkeywords={git, pencil code, secure shell}
}
\begin{document}

\maketitle
\thispagestyle{empty}

\section{Preparation}
\label{sec-1}


\subsection{\href{https://help.github.com/articles/generating-an-ssh-key/}{SSH key}}
\label{sec-1-1}

[You can work without an SSH key.
In that case you check out with
\texttt{‘git clone https://github.com/pencil-code/pencil-code.git’}.
The downside is that you will be asked for your Github password each
time you push a commit to the server.]

\begin{enumerate}
\item Create a dedicated SSH key for Github\footnote{Instead of using a dedicated key for Github, you could reuse another
SSH key, like the one you are using for logging into servers.
However, logging in to your servers has nothing to do with using Github, so
it is better practice to keep those two operations completely separate.}:

\lstset{frame=single,basicstyle=\ttfamily\footnotesize,language=sh,label= ,caption= ,captionpos=b,numbers=none}
\begin{lstlisting}
ssh-keygen -f ~/.ssh/github-key
\end{lstlisting}

You will be asked for a passphrase.
Depending on your usual practice, you can enter a passphrase or
leave it empty.\footnote{You can later add a passphrase to the key, or change an existing
passphrase using \texttt{‘ssh-keygen -f \textasciitilde{}/.ssh/github-key -p’}.}

\item Upload the public key to Github 
\begin{itemize}
\item Log in at Github and navigate to the \href{https://github.com/settings/profile}{settings page}
\item Select ‘SSH and GPG keys’
\item Click ‘New SSH key’
\item Upload the key \texttt{\textasciitilde{}/.ssh/github-key.pub}
\end{itemize}
\end{enumerate}


\section{Day-to-day usage}
\label{sec-2}

\subsection{Add SSH key to your SSH agent}
\label{sec-2-1}

On you local laptop or desktop computer, do
\lstset{frame=single,basicstyle=\ttfamily\footnotesize,language=sh,label= ,caption= ,captionpos=b,numbers=none}
\begin{lstlisting}
ssh-add ~/.ssh/github-key
\end{lstlisting}

If you work on a remote server, add the key locally before you connect
to the remote server, then do\footnote{If you are tired of explicitly typing the \texttt{‘-A’} option, you can add
an equivalent configuration option to your \texttt{\textasciitilde{}/.ssh/config} file.}
\lstset{frame=single,basicstyle=\ttfamily\footnotesize,language=sh,label= ,caption= ,captionpos=b,numbers=none}
\begin{lstlisting}
ssh -A beskow
\end{lstlisting}
[replace \emph{beskow} by the name of the server] to connect.

You can use the command \texttt{‘ssh-add -l’} to see which keys you currently
have added to the SSH agent [this also works on the remote server].


\subsection{Check out the Pencil Code}
\label{sec-2-2}

\lstset{frame=single,basicstyle=\ttfamily\footnotesize,language=sh,label= ,caption= ,captionpos=b,numbers=none}
\begin{lstlisting}
git clone git@github.com:pencil-code/pencil-code.git
\end{lstlisting}
This will create a directory \texttt{pencil-code}.


\subsection{Configure Git}
\label{sec-2-3}

Tell Git your user name and email:
\lstset{frame=single,basicstyle=\ttfamily\footnotesize,language=sh,label= ,caption= ,captionpos=b,numbers=none}
\begin{lstlisting}
git config user.name 'MY NAME'
git config user.email 'MY@EMAIL'
\end{lstlisting}

Recommended:
\lstset{frame=single,basicstyle=\ttfamily\footnotesize,language=sh,label= ,caption= ,captionpos=b,numbers=none}
\begin{lstlisting}
git config rebase.autoStash true
\end{lstlisting}
This will automatically stash away and later restore uncommitted
changes when they could interfere with some Git operations.


\subsection{Edit–commit cycle}
\label{sec-2-4}

\begin{enumerate}
\item Edit file(s)

\item Commit changes:
\lstset{frame=single,basicstyle=\ttfamily\footnotesize,language=sh,label= ,caption= ,captionpos=b,numbers=none}
\begin{lstlisting}
# Commit all changes to specific files: 
git commit <files>
# Alternatively, commit all changes to files known to Git:
git commit -a
\end{lstlisting}

An editor will pop up for you to enter the log message.
You can use \texttt{‘git commit [...] -m "MESSAGE"’} to set the log message
directly from the command line.

[To split existing changes into logical groups, you can use
\lstset{frame=single,basicstyle=\ttfamily\footnotesize,language=sh,label= ,caption= ,captionpos=b,numbers=none}
\begin{lstlisting}
git commit -p
\end{lstlisting}
to be asked for each block of changes (‘hunk’) whether you want to
commit it or not.]

\item To add any new files, use
\lstset{frame=single,basicstyle=\ttfamily\footnotesize,language=sh,label= ,caption= ,captionpos=b,numbers=none}
\begin{lstlisting}
git add <file(s)>
\end{lstlisting}

\item To see the current status:
\lstset{frame=single,basicstyle=\ttfamily\footnotesize,language=sh,label= ,caption= ,captionpos=b,numbers=none}
\begin{lstlisting}
git status
\end{lstlisting}

For more details, like history, etc., use
\lstset{frame=single,basicstyle=\ttfamily\footnotesize,language=sh,label= ,caption= ,captionpos=b,numbers=none}
\begin{lstlisting}
gitk --all     # graphical front-end; highly recommended
tig --all      # curses-based [may require 'apt install tig']
git log --all  # plain ASCII
\end{lstlisting}

The small colored circles in \emph{gitk} have the following meaning:
\begin{description}
\item[{Red}] Unregistered changes to files known to Git.
\item[{Green}] Changes added to the \emph{index} \footnote{The \emph{index} doesn't exist in Subversion.
If it gets in your way, use \texttt{‘git reset’} to turn changes from the
index into unregistered changes.}, but not yet committed.
\item[{Yellow}] The current HEAD commit.
\item[{Blue}] All other commits.
\end{description}

\item Go back to step 1.
\end{enumerate}


\subsection{Push commits to the server}
\label{sec-2-5}

To bring your commits to the server, use\footnote{The full form of these commands is\\
\noindent\mbox{}\qquad\qquad  \texttt{git pull -{}-rebase origin master}\\
\noindent\mbox{}\qquad\qquad \texttt{\# [Make sure whatever you just pulled didn't break anything ...]}\\
\noindent\mbox{}\qquad\qquad \texttt{git push origin master}\\
But you can (and should) configure Git such that you don't need to specify
the \emph{remote} (= origin), or the \emph{branch} (= master).}
\lstset{frame=single,basicstyle=\ttfamily\footnotesize,language=sh,label= ,caption= ,captionpos=b,numbers=none}
\begin{lstlisting}
git pull --rebase
# [Make sure whatever you just pulled didn't break anything ...]
git push
\end{lstlisting}
Running \texttt{‘git pull -{}-rebase’} prior to any \emph{push} operation saves you
some headache in case there are incoming changes from the server.


\subsection{Translation table Subversion \(\leftrightarrow\) Git}
\label{sec-2-6}

\begin{center}
\begin{tabular}{lll}
\toprule
\emph{Operation} & \emph{Subversion} & \emph{Git}\\
\midrule
Check out & \texttt{svn copy} & \texttt{git clone}\\
\midrule
Status & \texttt{svn status} & \texttt{git status}\\
\midrule
Diff & \texttt{svn diff} & \texttt{git diff}\\
\midrule
Add file & \texttt{svn add} & \texttt{git add}\\
\midrule
Commit and push & \texttt{svn commit} & \texttt{git commit}\\
 &  & \texttt{git push}\\
\midrule
Commit and push safely & \texttt{svn update} & \texttt{git commit}\\
 & \texttt{svn commit} & \texttt{git pull -{}-rebase}\\
 &  & \texttt{git push}\\
\bottomrule
\end{tabular}
\end{center}


\section{Troubleshooting}
\label{sec-3}

\subsection{Conflicts}
\label{sec-3-1}

Conflicts in files are marked up as in CVS or Subversion.

After you have resolved them in the file(s), run
\lstset{frame=single,basicstyle=\ttfamily\footnotesize,language=sh,label= ,caption= ,captionpos=b,numbers=none}
\begin{lstlisting}
git add <file(s)>
\end{lstlisting}
to tell Git that they are fixed.


\subsection{Have I lost data?}
\label{sec-3-2}

Most likely not. See Section 3 (“Don't panic”) of \emph{Git Best Practices} under
\texttt{\$PENCIL\_HOME/doc/git/}.


\subsection{I want to discard all of my local changes}
\label{sec-3-3}

[The following is one of the few commands that do lose data, but you
ware asking for exactly this]:
\lstset{frame=single,basicstyle=\ttfamily\footnotesize,language=sh,label= ,caption= ,captionpos=b,numbers=none}
\begin{lstlisting}
git reset --hard '@{upstream}'
git checkout .
\end{lstlisting}
\end{document}
